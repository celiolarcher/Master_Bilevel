\documentclass[conference]{IEEEtran}

\ifCLASSINFOpdf
  \usepackage[pdftex]{graphicx}
  \graphicspath{figures/}
  % \DeclareGraphicsExtensions{.pdf,.jpeg,.png}
\else
  % \usepackage[dvips]{graphicx}
  % \graphicspath{{../eps/}}
  % \DeclareGraphicsExtensions{.eps}
\fi

\usepackage[cmex10]{amsmath}
\usepackage{algorithmic}
\usepackage[latin1]{inputenc}
\usepackage[ruled,vlined,resetcount,linesnumbered]{algorithm2e}
%\usepackage[tight,footnotesize]{subfigure}

%\usepackage{cite}
\usepackage[nospace, noadjust]{cite}

\usepackage{supertabular}

%\hyphenation{op-tical net-works semi-conduc-tor}
\hyphenpenalty = 100000




\begin{document}

\title{Differential Evolution for Bilevel Programming}


%\author{
%\IEEEauthorblockN{Jaqueline S. Angelo}
%\IEEEauthorblockA{Laborat\'orio Nacional de Computa\c c\~ao Cient\'ifica,\\ Petr\'opolis - RJ, Brazil}
%\and
%\IEEEauthorblockN{Eduardo Krempser}
%\IEEEauthorblockA{Laborat\'orio Nacional de Computa\c c\~ao Cient\'ifica,\\ Petr\'opolis - RJ, Brazil}
%\and
%\IEEEauthorblockN{Helio J. C. Barbosa}
%\IEEEauthorblockA{Laborat\'orio Nacional de Computa\c c\~ao Cient\'ifica,\\ Petr\'opolis - RJ, Brazil}
%}
\author{\IEEEauthorblockN{Jaqueline S. Angelo\IEEEauthorrefmark{1},
Eduardo Krempser\IEEEauthorrefmark{1}\IEEEauthorrefmark{2},
Helio J.C. Barbosa\IEEEauthorrefmark{1}\IEEEauthorrefmark{3}}
\IEEEauthorblockA{
	\IEEEauthorrefmark{1}Laborat\'orio Nacional de Computa\c c\~ao Cient\'ifica,
Petr\'opolis - RJ, Brazil\\}
\IEEEauthorblockA{\IEEEauthorrefmark{2}Faculdade de Educa\c c\~ao Tecnol\'ogica do Estado do Rio de Janeiro (FAETERJ-Petr\'opolis)}
\IEEEauthorblockA{\IEEEauthorrefmark{3}Universidade Federal de Juiz de Fora, Juiz de Fora, MG, Brazil\\
Email: \{jsangelo, krempser, hcbm\}@lncc.br}
}

% conference papers do not typically use \thanks and this command
% is locked out in conference mode. If really needed, such as for
% the acknowledgment of grants, issue a \IEEEoverridecommandlockouts
% after \documentclass

% for over three affiliations, or if they all won't fit within the width
% of the page, use this alternative format:
% 
%\author{\IEEEauthorblockN{Michael Shell\IEEEauthorrefmark{1},
%Homer Simpson\IEEEauthorrefmark{2},
%James Kirk\IEEEauthorrefmark{3}, 
%Montgomery Scott\IEEEauthorrefmark{3} and
%Eldon Tyrell\IEEEauthorrefmark{4}}
%\IEEEauthorblockA{\IEEEauthorrefmark{1}School of Electrical and Computer Engineering\\
%Georgia Institute of Technology,
%Atlanta, Georgia 30332--0250\\ Email: see http://www.michaelshell.org/contact.html}
%\IEEEauthorblockA{\IEEEauthorrefmark{2}Twentieth Century Fox, Springfield, USA\\
%Email: homer@thesimpsons.com}
%\IEEEauthorblockA{\IEEEauthorrefmark{3}Starfleet Academy, San Francisco, California 96678-2391\\
%Telephone: (800) 555--1212, Fax: (888) 555--1212}
%\IEEEauthorblockA{\IEEEauthorrefmark{4}Tyrell Inc., 123 Replicant Street, Los Angeles, California 90210--4321}}

% use for special paper notices
%\IEEEspecialpapernotice{(Invited Paper)}


\maketitle

%--------------------------------------------------------------------
\begin{abstract}
%--------------------------------------------------------------------
%\boldmath


Bilevel programming problems are characterized by two optimization 
problems which are hierarchically related, where
to each feasible upper level solution an optimal solution 
in the lower level problem must be associated.
%
These problems  appear in many practical applications, and 
a variety of solution techniques can be found in the literature.
%
In this paper, an algorithm is proposed which uses differential evolution  to solve
both the upper and lower level optimization problems.
%
Several test problems from the literature are solved in order to assess the performance
of the proposed method.


\end{abstract}


% no keywords

%--------------------------------------------------------------------
\section{Introduction}
%--------------------------------------------------------------------

The Bilevel programming problem (BLP) is characterized as an optimization 
problem that has in its constraints another optimization problem.
% 
In this hierarchical structure each level of the hierarchy seeks to 
optimize an objective function, controlling a set of variables subject
to certain constraints. 
%
One can think of a leader (upper level) that optimizes an objective 
function considering both his constraints and the reaction of the 
follower (lower level), which in turn makes his decision to optimize 
its objective function taking into account what has been imposed by 
the leader.


Due to this hierarchical structure, even bilevel problems where all functions
involved are linear are, in general, quite difficult to be solved, and were shown to be NP-hard by 
Hansen et. al \cite{Hansen92}.
%
A variety of exact methods have been developed,  involving the use of enumerative schemes,
application of a penalty function, or gradient methods, among others.
%
The most used exact method investigate 
the properties of the so-called inducible region to obtain (necessary 
and sufficient) conditions to replace the follower's problem with
its Karush-Kunh-Tucker (KKT) conditions, appending the resultant system
to the leader's problem, transforming the bilevel problem into a 
standard single-level programming problem.
%
However, those approaches assume (explicit) knowledge of the 
analytical form of the objective function and the constraints of 
the problem and, sometimes, the transformed problem is not
equivalent to the original bilevel problem.


The investigations in multilevel programming problems are strongly 
motivated by real-world applications found in economics, engineering, 
transportation, networks, planning, etc \cite{Dempe03}.
%
Due to the complexity of those applications, intelligent heuristics
such as evolutionary computation, become attractive and 
even enable the resolution of such problems.
%
Although these techniques cannot guarantee to find the optimal 
solution, but only a good approximation to it, they can help overcome 
the challenges of bilevel programming problems, such as nonconvexity and 
nondifferentiability, large number of variables and/or constraints,
and non-unique optimal solution for the follower's problem.
%
Heuristic methods, can also deal with cases where it is not possible 
to make strong assumptions about the objective function and/or constraints of the 
problem. Those cases, known as ``black box'' optimization,  occur often in practice, for example, 
when the computation of the objective functions and/or constraints  require a potentially expensive
computer simulation.


Differential Evolution (DE) is an evolutionary algorithm, 
proposed by Storn and Price (1997), aimed at solving continuous 
optimization problems. 
%
It is a simple, stochastic, population-based technique that was 
initially formulated to deal with unconstrained problems, but
which can be efficiently applied to constrained problems 
\cite{Suganthan11,Krempser11}.
%
In this paper, a DE algorithm (BlDE) for solving general BLPs  is proposed,
and its performance is assessed using a collection of well known 
test-problems  from the literature.
%
%Each level of the BLP is solved by a different population, trying
%to find the optimal solution for the test problems considered.


The paper is organized as follows: Section \ref{sec:BLP} describes
the structure of a general bilevel optimization problem; Section
\ref{sec:review} presents a brief literature review on related
work emphasizing some evolutionary methods already proposed.
The DE technique is summarized
in Section \ref{sec:DE}, while in Section \ref{sec:BlDE} the
proposed solution method BlDE is presented.
The description of each test problem and the results of the computational experiments 
are presented in Section \ref{sec:experiments}; the
conclusions are left for Section \ref{sec:conclusion}.



%--------------------------------------------------------------------
\section{The Bilevel Programming Problem}\label{sec:BLP}
%--------------------------------------------------------------------

The bilevel programming problem was initially proposed by
the economist Von Stackelberg in 1934 to describe oligopoly 
situations in economy markets.
%
The Stackelberg's model describes a competitive game between two 
companies that make decisions sequentially.


In the BLP, two decision makers, the leader in the upper level and the 
follower in the lower level, are hierarchically related, where
the  leader's decisions affect both the follower's payoff 
function and allowable actions, and vice-versa.
%
Each decision maker has control over a set of variables, seeking 
to optimize its own objective function.
%
Once the leader chooses the value of its variables, the follower reacts 
by choosing the value of its own variables in order to optimize its 
objective function.


The bilevel programming problem can be written as:
%
\begin{equation}\label{equ:bilevel}
\begin{array}{lll}
	(L) & \min_{x \in X}   & f_{1}(x,y(x)) \\
		  & \text{subject to} & g_{1}(x,y(x)) \leq 0 \\ 
		  & \hfill (F)     & y(x) \in {\text{arg}\min}_{y \in Y} f_{2}(x,y) \\
		  &                  & \text{subject to} \quad g_{2}(x,y) \leq 0
\end{array}
\end{equation}
%
The higher level decision maker ($L$) --the leader-- has control over 
the $x$ variables, and makes his decision first, fixing $x$.
%
The lower level decision maker ($F$) --the follower-- controls the $y$ variables. 
Reacting to the decision of the leader, the $y$ variables are set in response to the given $x$.

In the experiments performed here, it is assumed that $X = R^{n_1}$
and $Y = R^{n_2}$, $n_1$ and $n_2$ being the number of upper and lower level
variables, respectively.

In problem (\ref{equ:bilevel}) the reaction set of the follower, 
namely $R(x)$ such that $y(x) \in R(x)$, defines the follower's response 
given an $x$ fixed by the leader.
%
One difficulty that arises in solving BLP is that, if $R(x)$ is not 
single-valued for all possible $x$, the leader may not achieve 
his minimum payoff, since the follower has multiple minimum 
solutions to choose.
%
In this case, there is no guarantee that the follower's choice is the 
best for the leader, leading to sub-optimal solutions in the 
leader's problem.


Another challenge lies in the fact that unless a solution is optimal 
for the lower level problem, it cannot be feasible for the overall 
problem.
%
This suggests that approximate optimization methods
cannot be used to solve the lower level problem, as they do not guarantee 
that the optimal solution is actually found.
%
However, from the practical point of view near-optimal solutions are 
often acceptable \cite{Deb09}, especially when the lower level 
problem is too costly to be exactly solved thus rendering the use of exact methods impractical.


%Next section, a brief overview of existing methods for solving BLP
%are presented.


%--------------------------------------------------------------------
\section{Literature Review}\label{sec:review}
%--------------------------------------------------------------------

There have been numerous  methods proposed for solving bilevel
programming problems.
%
Most of them are specialized to the linear case, where all functions 
involved are linear.
%
In general, for solving the BLP one can find three different approaches: 
enumerative schemes, mainly for the linear BLP 
\cite{Candler82, Bialas82, Onal93, Campelo00}, 
penalty methods \cite{Shimizu81, Anandalingam90, Ishizuka92, Zheng12},
and methods based on the KKT conditions
\cite{FortunyAmat81, Bard90, Hansen92, Judice92, Shi06}.
%
All of these methods are very time consuming, especially 
when dealing with large sized problems.
%
For this reason, the use of intelligent heuristics is growing, as they are capable of finding good solutions 
for complex single-level optimization problems in reasonable computer times.


Mathieu et al. \cite{Mathieu94} firstly developed a genetic algorithm (GA) to solve linear BLP. 
%
In their proposal, the leader's decision variables are generated using
a modified GA, which only used the mutation operation, while the follower's decision 
variables are obtained using an algorithm based on the simplex method.
%
The algorithm proposed presented better results when compared to 
Bard's grid search technique \cite{Bard83} in randomly generated test-problems
with different sizes.
%
However, the time consumed was higher, varying between 24.82s and 
3281.59s against 4.23 and 63.99 provided by the grid search technique.
%
%The authors pointed that the reasons for the slowness could be related
%to the fact that they did not exploit crossover and that the stopping 
%criteria was more stringent than for the grid search method.
%
The same happens with a simulated annealing algorithm (SA) when applied to
the same test-problems \cite{Anandalingam89}.
%
SA presented better results than the grid search technique--however,
worse than those produced by the GA--in a higher computational time.


Hejazi et al. \cite{Hejazi02} proposed a GA for the linear BLP.
%
The second level problem was replaced by the KKT optimality 
conditions yielding a single-level optimization problem which was 
solved by the GA.
%
The proposed method was compared to a hybrid tabu-ascent algorithm
\cite{Gendreau96} in randomly generated problems of different sizes.
%
The proposed GA was faster than the tabu-ascent algorithm in all 
test problems, except for one.
%
However, for the largest problems, the quality of the solutions obtained
by the tabu-ascent algorithm was superior to that achieved by the GA proposed.



Oduguwa and Roy \cite{Oduguwa02} proposed a coevolutionary GA for
solving the general BLP.
%
The proposed method implemented two populations, one to solve the 
leader's problem, and another to solve the follower's problem.
%
A coevolutionary operator synchronizes one population with the 
fittest members of the other population.
%
Hence, the leader's variables $x$ are optimized with a subset of 
the best follower's variables, and the follower's variables $y$ 
are optimized with all $x$ variables fixed.
%
An additional elite population was maintained to identify the 
elite members from both populations, and was updated in every
generation with the best members from the current generation.
%
The algorithm proposed presented similar solutions when compared 
to different methods for solving four test problems available in 
literature \cite{Bard98}.
%
%Computational time was not reported.


In \cite{Zhu06}, Zhu et. al proposed a DE algorithm combined with
an interior point method for solving nonlinear problems with linear
constraints.
%----- (DE/best/2/exp) Pop=10, F1=F2=0.8, CR=0.9, MaxG=100.
%
The DE algorithm was designed to operate in the leader's problem
while the interior point method was responsible for solving the 
follower's problem.
%
The DE was used to generate and modify the leader's variables.
%
For each individual with a fixed $x$, $y$ is obtained by solving
the follower's problem using the interior point method.
%
The algorithm was capable of finding the same results reported in the
original references for the six problems tested.


The DE was also used by Koh \cite{Koh07} to solve a transportation 
BLP, where a gradient based algorithm was used to solve the lower
level problem.
%----- (DE/best/1/bin) Pop=20, F=0.8, CR=0.8, MaxG=40, runs=30.
%
The algorithm is similar to the one proposed in \cite{Zhu06}: 
the DE operates in the upper level, evolving $x$, while the gradient
based method generates the optimal $y$ for each $x$ fixed by the 
upper level problem.
%
Seven BLPs taken from the literature, as well as specific applications
in transportation problems, were used to assess the quality of the
algorithm proposed.
%
The algorithm was capable of finding  solutions identical to the ones
reported in the original references for most test problem, and provided
better solutions in two of them.
%
For the transportation problems tested the algorithm generated
competitive results.




%--------------------------------------------------------------------
\section{Differential Evolution}\label{sec:DE}
%--------------------------------------------------------------------

DE was originally proposed by Storm and Price \cite{Storn97} and has been shown 
to be a simple and effective algorithm for global optimization, specially for continuous variables.
%A pseudo-code is presented in Algorithm \ref{alg:de}. 

The basic operation performed is the addition to each design variable in a given candidate
solution of a term which is the scaled difference between the values of such variable in other
candidate solutions in the population. 
The number of differences applied, the way in which the individuals are selected, and the 
distribution of recombination determine the DE variant.
%
As with other evolutionary algorithms, DE performance is influenced by its 
particular parameters, namely, the chosen variant, and the scale factor ($F$) adopted. 

%\rever{
%Here, the basic mutation operator of the DE (DE/rand/1/bin) was applied. 
%This variant  select randomly the individuals in the population, leading to:
%\begin{equation}
% 	u_{i,j,G+1} = x_{r_1, j, G} + F.(x_{r_2, j, G}-x_{r_3, j, G})
%\end{equation}
%where $r_1, r_2$ and $r_3$ are randomly selected individuals. 
%}


Here, the DE variant DE/target-to-rand/1/bin was adopted. 
%
In preliminary experiments this strategy presented the best results to 
the overall test problems considered in this paper.
%(DE/mecanismo-de-selecao/numero-de-diferencas/modelo-de-recombinacao)
%
This variant uses randomly selected individuals ($r_1, r_2, r_3$) of the population
and the target individual $x_i$ (the one that will be used in the comparison after
the mutation, also called current individual), leading to:
%
\begin{equation}\nonumber
 	u_{i,j,G+1} = x_{i,j,G} + F.(x_{r_3,j,G}-x_{i,j,G}) + F.(x_{r_1,j,G}-x_{r_2,j,G})
\end{equation}

In addition, a crossover operation is performed, using the parameter CR.
%, as explained in Algorithm \ref{alg:de}. 
%
%\rever{
%We consider to $F$ and $CR$ parameters the values: $F=0.7$ and $CR=0.9$.
%}

For each design variable, lower and upper bounds are assumed, 
and whenever a given component $x_{i,j}$ of a candidate solution $x_i$ is generated outside its prescribed range,
a standard projection operation is performed:
\begin{eqnarray*} 
\mbox{If\ } x_{i,j} > x_{j}^U \mbox{\ then\ } x_{i,j} = x_{j}^U \mbox{\ and\ }
\mbox{if\ } x_{i,j} < x_{j}^L \mbox{\ then\ } x_{i,j} = x_{j}^L .
\end{eqnarray*}



%--------------------------------------------------------------------
\section{The Proposed Solution Method}\label{sec:BlDE}
%--------------------------------------------------------------------

In the proposed Bilevel Differential Evolution algorithm (BlDE), the 
lower level problem is solved for each upper level point.
%
The solution method uses a DE to generate and evolve the upper level 
variables. For each upper level point, another DE is used to solve the
lower level problem, by generating and evolving the lower level
variables.
%
Algorithms \ref{alg:de} and \ref{alg:deFollower} describe the proposed
method.
%
\begin{algorithm}[!ht]
 \caption{Algorithm DE Leader.}
 \label{alg:de}
 \SetKwData{POP}{np}\SetKwData{GEN}{gen}\SetKwData{F}{F}\SetKwData{CR}{CR}\SetKwData{G}{G}
 \SetKwData{N}{n}
 \SetKwFunction{Evaluate}{Evaluate $f_1$}\SetKwFunction{Create}{CreateRandomInitialPopulation}
 \SetKwFunction{Select}{SelectRandomly}
 \SetKwFunction{RandInt}{RandInt}
 \SetKwFunction{Rand}{Rand}
 \SetKwFunction{DEFollower}{DEFollower}
 \SetKwInOut{Input}{input}\SetKwInOut{Output}{output}
 \Input{\POP (population size), \GEN (\# of generations), \\ \F (mutation scaling), \CR (crossover rate)}
 \BlankLine
 \G$\leftarrow$ $0$\;
 \Create{\POP}\;
 \For{$i\leftarrow 1$ \KwTo \POP}{
  $\overrightarrow{y}$ = \DEFollower{npf, genf, F, CR, $\overrightarrow{x}_{i, G}$}\;
  \Evaluate{$\overrightarrow{x}_{i,\G}$, $\overrightarrow{y}$} \tcc*{$\overrightarrow{x}_{i,\G}$ is an individual in the population}
 }
 \For{\G$\leftarrow 1$ \KwTo \GEN}{
   \For{$i\leftarrow 1$ \KwTo \POP}{
      \Select{$r_1, r_2, r_3$} \tcc*{$r_1 \neq r_2 \neq r_3 \neq i$}
      $jRand \leftarrow$\RandInt{1, \N} \tcc*{\N is the number of variables of the leader problem}
      \For{$j\leftarrow 1$ \KwTo \N}{
        \eIf{
	  \Rand{$0, 1$} $<$ \CR or $j = jRand$
	}{ 
	  %$u_{i, j,G+1}$ = $x_{r_3,j,G} + F.(x_{r_1,j,G} - x_{r_2, j, G})$\;\label{lineVariant}
	   $u_{i,j,G+1}$ = $x_{i,j,G} + F.(x_{r_3,j,G}-x_{i,j,G}) + $\\
		   \quad\quad\quad\quad$F.(x_{r_1,j,G}-x_{r_2,j,G})$\;\label{lineVariant}
	}{
	  $u_{i, j, G+1}$ = $x_{i, j, G}$\;
	}
      }
      $\overrightarrow{y}$ = \DEFollower{npf, genf, F, CR, $\overrightarrow{u}_{i, G+1}$}\;
      \eIf{$ f_1(\overrightarrow{u}_{i, G+1}, \overrightarrow{y}) \leq f_1(\overrightarrow{x}_{i,G}, \overrightarrow{y})$}{
	$\overrightarrow{x}_{i,G+1}$ = $\overrightarrow{u}_{i,G+1}$\;
      }{
	$\overrightarrow{x}_{i,G+1}$ = $\overrightarrow{x}_{i,G}$\;
      }
   }
 }
\end{algorithm}
%
\begin{algorithm}[!ht]
 \caption{Algorithm DE Follower.}
 \label{alg:deFollower}
 \SetKwData{POP}{npf}\SetKwData{GEN}{genf}\SetKwData{F}{F}\SetKwData{CR}{CR}\SetKwData{G}{G}
 \SetKwData{N}{nf}
 \SetKwData{V}{v}
 \SetKwFunction{Evaluate}{Evaluate $f_2$}\SetKwFunction{Create}{CreateRandomInitialPopulation}
 \SetKwFunction{Select}{SelectRandomly}
 \SetKwFunction{RandInt}{RandInt}
 \SetKwFunction{Rand}{Rand}
 \SetKwFunction{SelectBest}{SelectBestIndividual}
 \SetKwInOut{Input}{input}\SetKwInOut{Output}{output}
 \Input{\POP (follower population size), \GEN (\# of generations to follower DE), \F (mutation scaling), \CR (crossover rate), $\overrightarrow{\V}$ (leader variables)}
 \BlankLine
 \G$\leftarrow$ $0$\;
 \Create{\POP}\;
 \For{$i\leftarrow 1$ \KwTo \POP}{
  \Evaluate{$\overrightarrow{\V}$, $\overrightarrow{x}_{i,\G}$} \tcc*{$\overrightarrow{x}_{i,\G}$ is an individual in the population}
 }
 \For{\G$\leftarrow 1$ \KwTo \GEN}{
   \For{$i\leftarrow 1$ \KwTo \POP}{
      \Select{$r_1, r_2, r_3$} \tcc*{$r_1 \neq r_2 \neq r_3 \neq i$}
      $jRand \leftarrow$\RandInt{1, \N} \tcc*{\N is the number of variables in the follower problem}
      \For{$j\leftarrow 1$ \KwTo \N}{
        \eIf{
	  \Rand{$0, 1$} $<$ \CR or $j = jRand$
	}{ 
	  %$u_{i, j,G+1}$ = $x_{r_3,j,G} + F.(x_{r_1,j,G} - x_{r_2, j, G})$\;\label{lineVariant}
	   $u_{i,j,G+1}$ = $x_{i,j,G} + F.(x_{r_3,j,G}-x_{i,j,G}) + $\\
		   \quad\quad\quad\quad$F.(x_{r_1,j,G}-x_{r_2,j,G})$\;
	}{
	  $u_{i, j, G+1}$ = $x_{i, j, G}$\;
	}
      }
      \eIf{$ f_2(\overrightarrow{\V}, \overrightarrow{u}_{i, G+1}) \leq f_2(\overrightarrow{\V}, \overrightarrow{x}_{i,G})$}{
	$\overrightarrow{x}_{i,G+1}$ = $\overrightarrow{u}_{i,G+1}$\;
      }{
	$\overrightarrow{x}_{i,G+1}$ = $\overrightarrow{x}_{i,G}$\;
      }
   }
 }
 \Return \SelectBest
\end{algorithm}

The main steps of the algorithm are summarized as follows:

{\it Step 0: Initialization.} The algorithm starts with a
population, of size $np$, of vectors containing the
upper level variables $x \in R^{n_1}$.
%
The upper level variables are initialized with random values 
and the lower level variables are determined by executing the 
lower level procedure (Algorithm \ref{alg:deFollower}), 
which generates the vector $y \in R^{n_2}$ of lower level variables.


{\it Step 1: Evolution at the upper level.} Following the basic 
DE algorithm described in Algorithm \ref{alg:de}, the upper level individuals  are 
mutated and recombined. 

{\it Step 2: Evaluation of each upper level individual.} 
To evaluate the individuals in the upper level, where fitness is 
assigned based on the upper level function and constraints,
the lower level procedure is performed.
The solution returned, that is, the best individual obtained in the 
lower level procedure, is used to evaluate the upper level individual.

{\it Step 3: Evolution at lower level.} For fixed upper level variables, 
a new DE algorithm is executed, as described in Algorithm \ref{alg:deFollower}.
The individuals are evaluated based on the lower level function 
and constraints. 
Finally, the procedure returns the best value of the lower level problem.


In order to handle the constraints, the method proposed by Deb in 
\cite{Deb00} was applied, which enforces the following criteria:
(i) any feasible solution is preferred to any infeasible solution,
(ii) among two feasible solutions, the one having better objective 
function value is preferred, and
(iii) among two infeasible solutions, the one having smaller constraint 
violation is preferred.


%--------------------------------------------------------------------
\section{Computational Experiments}\label{sec:experiments}
%--------------------------------------------------------------------


In this section, eighteen test-problems taken from different sources in the literature 
are used in order to assess the performance of the proposed algorithm.
%
The BlDE algorithm was also tested in a set of six unconstrained
test problems proposed in \cite{Sinha12}.
%
The test-collection (SMD1 to SMD6) in \cite{Sinha12} aims to induce difficulties at
both levels of the BLP, independently and collectively, such that the 
performance of the algorithms could be better evaluated when handling the two
levels.


For the BlDE algorithm, parameters $F$ and $CR$ were set to $F=0.7$ 
and $CR=0.9$. 
%
For all test-problems, the population size was set to 30 individuals 
and 30 independent runs were performed.

%--------------------------------------------------------------------
\subsection{Test problems}

The eighteen test problems initially considered are:
%
\begin{enumerate}
%
\footnotesize{

\item Shimizu and Aiyoshi (1981) \cite{Shimizu81}: % 1
%
\begin{equation}\nonumber
\begin{array}{l}
\min_{x} f_{1}(x,y) =  x^2 + (y - 10)^2 \\
\text{s.t. } \hspace{0.3cm} -x + y \leq 0 \\
	\hspace{0.8cm} x \in [0,15] \\
	\hspace{0.8cm} \min_{y} f_2 (x,y) = (x + 2y -30)^2 \\
	\hspace{0.8cm} \text{s.t. } \hspace{0.3cm} x + y - 20 \leq 0 \\
	\hspace{1.5cm} y \in [0,20]
%(x*, y*) = (10, 10). f1=100, f2=0
\end{array}
\end{equation}


\item Shimizu and Aiyoshi (1981) \cite{Shimizu81}: % 2
%
\begin{equation}\nonumber
\begin{array}{l}
\min_{x} f_{1}(x,y) = (x_1-30)^2 + (x_2-20)^2 -20y_1 + 20y_2 \\
\text{s.t. } \hspace{0.3cm} x_1+2x_2 \geq 30 \\
	\hspace{0.8cm} x_1+x_2 \leq 25 \\
	\hspace{0.8cm} x_2 \leq 15 \\
	\hspace{0.8cm} \min_{y} (x_1-y_1)^2 + (x_2-y_2)^2 \\
	\hspace{0.8cm} \text{s.t. } \hspace{0.3cm} y_1, y_2 \in [0,20] 
%(x*, y*) = (20,5,10,5). f1=225, f2=100 
\end{array}
\end{equation}


\item Candler and Townsley (1982) \cite{Candler82}: % 3
%
\begin{equation}\nonumber
\begin{array}{l}
\max_{x} f_{1}(x,y) = 8x_1 + 4x_2 - 4y_1 + 40y_2 + 4y_3  \\
\text{s.t. } \hspace{0.3cm} \max_{y} f_2 (x,y)  = -x_1 - 2x_2 - y_1 - y_2 - 2y_3 \\
	\hspace{0.8cm} \text{s.t } \hspace{0.3cm} -y_1 + y_2 + y_3 \leq 1 \\
		\hspace{1.6cm} 2x_1 - y_1 + 2y_2 - 0.5y_3 \leq 1 \\
    	\hspace{1.6cm} 2x_2 + 2y_1 - y_2 - 0.5y_3 \leq 1 \\
    	\hspace{1.6cm} x , y \geq 0
%(x*, y*) = (0, 0.6, 0.4, 0, 0.9). f1=29.2, f2=-3.2
\end{array}
\end{equation}


\item Bard and Falk (1982) \cite{Bard82}: % 4
%
\begin{equation}\nonumber
\begin{array}{l}
\max_{x} f_{1}(x,y) = 2x_1 - x_2 - 0.5y_1 \\
\text{s.t. } \hspace{0.3cm} \max_{y} f_2 (x,y) = -x_1 - x_2 + 4y_1 - y_2\\
	\hspace{0.8cm} \text{s.t. } \hspace{0.3cm} 2x_1 - y_1 + y_2 \geq 2.5 \\
	    \hspace{1.6cm} -x_1 + 3x_2 - y_2 \geq -2 \\
    	\hspace{1.6cm} -x_1 - x_2 \geq -2 \\
    	\hspace{1.6cm} x,y \geq 0
%(x*, y*) = (1, 0, 0.5, 1). f1=1.75, f2=0
\end{array}
\end{equation}


\item Aiyoshi and Shimizu (1984) \cite{Aiyoshi84}: % 5
%
\begin{equation}\nonumber
\begin{array}{l}
\min_{x} f_{1}(x,y) = 2x_1 + 2x_2 - 3y_1 - 3y_2 - 60 \\
\text{s.t. } \hspace{0.3cm} x_1 + x_2 + y_1 - 2y_2 - 40 \leq 0 \\
	\hspace{0.8cm} \min_{y} f_2 (x,y) = (y_1 - x_1 + 20)^2 + (y_2 - x_2 + 20)^2\\
	\hspace{0.8cm} \text{s.t. } \hspace{0.3cm}  2y_1 - x_1 + 10 \leq 0 \\
		\hspace{1.6cm} 2y_2 - x_2 + 10 \leq 0 \\
		\hspace{1.6cm} x \in [0,50], y \in [-10,20]
%(x*, y*) = (25, 30, 5, 10). f1=5, f2=0
\end{array}
\end{equation}


\item Bard (1988) \cite{Bard88}: % 6
%
\begin{equation}\nonumber
\begin{array}{l}
\min_{x} f_{1}(x,y) = (x-5)^2 + (2y+1)^2 \\
\text{s.t. } \hspace{0.3cm} \min_{y} f_2 (x,y) = (y-1)^2 -1.5xy \\
	\hspace{0.8cm} \text{s.t. } \hspace{0.3cm} 3x-y \geq 3 \\
	    \hspace{1.6cm} -x+0.5y \geq -4 \\
    	\hspace{1.6cm} -x-y \geq -7 \\
      	\hspace{1.6cm} x,y \geq 0
%(x*, y*) = (1.778 , 2.333). f1=42.5, f2=-4.445
\end{array}
\end{equation}


\item Bard (1988) \cite{Bard88}: % 7
%
\begin{equation}\nonumber
\begin{array}{l}
\min_{x} f_{1}(x,y) = -x_1^2 -3x_2 -4y_1 + y_2^2 \\
\text{s.t. } \hspace{0.3cm} x_1^2 +2x_2 \leq 4 \\
	\hspace{0.8cm} \min_{y} f_2 (x,y) = 2x_1^2 +y_1^2 -5y_2 \\
	\hspace{0.8cm} \text{s.t. } \hspace{0.3cm}  x_1^2 -2x_1 +x_2^2 -2y_1 + y_2 \geq 3 \\
		\hspace{1.6cm} x_2 +3y_1 -4y_2 \geq 4 \\
		\hspace{1.6cm} x,y \geq 0
%(x*, y*) = (0,2,4,1). f1=-21, f2=11
\end{array}
\end{equation}


\item Anandalingam and White (1990) \cite{Anandalingam90}: % 8
%
\begin{equation}\nonumber
\begin{array}{l}
\max_{x} f_{1}(x,y) = x + 3y \\
\text{s.t. } \hspace{0.3cm} \max_{y} f_2 (x,y) = x - 3y \\
	\hspace{0.8cm} \text{s.t. } \hspace{0.3cm}  -x - 2y \leq -10 \\
	    \hspace{1.6cm} x - 2y \leq 6  \\
    	\hspace{1.6cm} 2x - y \leq 21  \\
    	\hspace{1.6cm} x + 2y \leq 38 \\
    	\hspace{1.6cm} -x + 2y \leq 18 \\
		\hspace{1.6cm} x,y \geq 0
%(x*, y*) = (16, 11). f1=49, f2=-17
\end{array}
\end{equation}


\item Savard and Gauvin (1994) \cite{Savard94}: % 9
%
\begin{equation}\nonumber
\begin{array}{l}
\min_{x} f_{1}(x,y) = (x-1)^2 + 2y_{1}^{2} - 2x \\
\text{s.t. } \hspace{0.3cm} \min_{y} f_2 (x,y) = (2y_1-4)^2 + (2y_2-1)^2 + xy_1 \\
	\hspace{0.8cm} \text{s.t. } \hspace{0.3cm} 4x + 5y_1 + 4y_2 \leq 12 \\
	    \hspace{1.6cm} -4x - 5y_1 + 4y_2 \leq -4 \\
    	\hspace{1.6cm} 4x - 4y_1 + 5y_2 \leq 4 \\
	    \hspace{1.6cm} -4x + 4y_1 + 5y_2 \leq 4 \\
	    \hspace{1.6cm} x,y \geq 0
%(x*, y*) = (1.889, 0.889, 0). f1=-1.21, f2=7.61
\end{array}
\end{equation}


\item Falk and Liu (1995) \cite{Falk95}: % 10
%
\begin{equation}\nonumber
\begin{array}{l}
\min_{x} f_{1}(x,y) = x_1^2 -2x_1 +x_2^2 -2x_2 +y_1^2 + y_2^2\\
\text{s.t. } \hspace{0.3cm} \min_{y} f_2 (x,y) = (y_1 -x_1)^2 + (y_2 -x_2)^2 \\
	\hspace{0.8cm} \text{s.t. } \hspace{0.3cm} x \geq 0, y \in [0.5,1.5]
%(x*, y*) = (.5, .5, .5, .5). f1=-1, f2=0
\end{array}
\end{equation}


\item Shimizu and Lu (1995) \cite{Shimizu95} % 11
%
\begin{equation}\nonumber
\begin{array}{l}
\min_{x} f_{1}(x,y) = 16x^2 + 9y^2 \\
\text{s.t. }  \hspace{0.3cm} -4x + y \leq 0 \\ 
	\hspace{0.8cm} \min_{y} f_2 (x,y) = (x + y -10)^4\\
	\hspace{0.8cm} \text{s.t. }  \hspace{0.3cm} 4x + y - 50 \leq 0 \\
		\hspace{1.6cm} x,y \geq 0 
%(x*, y*) = (11.25,5). f1=2250, f2=197.75
\end{array}
\end{equation}


\item Bard (1998) \cite{Bard98}: % 12 (pag. 197)
%
\begin{equation}\nonumber
\begin{array}{l}
\min_{x} f_{1}(x,y) = x-4y \\
\text{s.t. } \hspace{0.3cm} \min_{y} f_2 (y)  = y \\
	\hspace{0.8cm} \text{s.t. } \hspace{0.3cm} -x-y \leq -3 \\
	    \hspace{1.6cm} -2x + y \leq 0 \\
    	\hspace{1.6cm} 2x + y \leq 12 \\
	    \hspace{1.6cm} -3x + 2y \geq -4 \\
    	\hspace{1.6cm} x,y \geq 0
%(x*, y*) = (4,4). 
\end{array}
\end{equation}


\item Bard (1998) \cite{Bard98}: % 13 (pag. 263) 
%
\begin{equation}\nonumber
\begin{array}{l}
\min_{x} f_{1}(x,y) = x + y \\
\text{s.t. }  \hspace{0.3cm} \min_{y} f_2 (x,y) = -5x -y \\
	\hspace{0.8cm} \text{s.t. }  \hspace{0.3cm} -x-y/2 \leq -2 \\
		\hspace{1.6cm} -x/4 + y \leq 2 \\
		\hspace{1.6cm}  x+y/2 \leq 8 \\
		\hspace{1.6cm}  x -2y \leq 4 \\
		\hspace{1.6cm} x,y \geq 0
%(x*, y*) = (0.888,2.222). f1=3.111, f2=-6.662
\end{array}
\end{equation}


\item Bard (1998) \cite{Bard98}: % 14 (pag. 306)
%
\begin{equation}\nonumber
\begin{array}{l}
\min_{x} f_{1}(x,y) = (x-1)^2 + (y-1)^2  \\
\text{s.t. } \hspace{0.3cm} \min_{y} f_2 (x,y) = 0.5y^2 + 500y -50xy \\
       	\hspace{0.8cm} x,y \geq 0
%(x^{*}, y^{*}) = (10.02,0.82). f1=81.33, f2=-0.4838
\end{array}
\end{equation}


\item Oduguwa and Roy (2002) \cite{Oduguwa02}: % 15 (Ex1) $P(x) \neq 1$
%
\begin{equation}\nonumber
\begin{array}{l}
\max_{x} f_{1}(x,y) = 100x + 1000y_1 \\
\text{s.t. } \hspace{0.3cm} \max_{y} f_2 (x,y) = y_1 + y_2\\
	\hspace{0.8cm} \text{s.t. } \hspace{0.3cm}  x + y_1 - y_2 \leq 1\\
	\hspace{1.6cm} y_1 + y_2 \leq 1 \\
	\hspace{1.6cm}	x \in [0,1], y \in [0,1]
%(x*, y*) = (?,?). f1=1000, f2=1
\end{array}
\end{equation}


\item Rajesh et. al (2003) \cite{Rajesh03}: % 16 (Ex2)
%
\begin{equation}\nonumber
\begin{array}{l}
\min_{x} f_{1}(x,y) = (x-3)^2 + (y-2)^2 \\
\text{s.t. }  \hspace{0.3cm} \min_{y} f_2 (x,y) = (y-5)^2 \\
	\hspace{0.8cm} \text{s.t. } \hspace{0.3cm}  2x-y \geq -1 \\
		\hspace{1.6cm} -x+2y \geq 2 \\
		\hspace{1.6cm} -x-2y \geq -14 \\		
		\hspace{1.6cm} x \in [0,8], y \geq 0
%(x*, y*) = (1,3). f1=5, f2=4.
\end{array}
\end{equation}


\item Rajesh et. al (2003) \cite{Rajesh03}: % 17 (Ex3)
%
\begin{equation}\nonumber
\begin{array}{l}
\min_{x} f_{1}(x,y) = (x-3)^2 + (y-2)^2 \\
\text{s.t. } \hspace{0.3cm} 2x-y \geq -1 \\
	\hspace{0.8cm} -x+2y \geq 2 \\
	\hspace{0.8cm} -x-2y \geq -14 \\	
	\hspace{0.8cm} \text{s.t. } \hspace{0.3cm} \min_{y} f_2 (x,y) = (y-5)^2 \\
		\hspace{1.6cm} x \in [0,8], y \geq 0
%(x*, y*) = (3,5). f1=9
\end{array}
\end{equation}


\item Rajesh et. al (2003) \cite{Rajesh03}: % 18 (Ex4)
%
\begin{equation}\nonumber
\begin{array}{l}
\max_{x} f_{1}(x,y) = -2x+11y \\
\text{s.t. }  \hspace{0.3cm} \max_{y} f_2 (x,y) = -x-3y \\
	\hspace{0.8cm} \text{s.t. }  \hspace{0.3cm} x-2y \leq 4 \\
		\hspace{1.6cm} 2x-y \leq 24 \\
		\hspace{1.6cm} 3x+4y \leq 96 \\
		\hspace{1.6cm} x+4y \leq 126 \\			
		\hspace{1.6cm} -4x+5y \leq 65 \\
		\hspace{1.6cm} x+4y \geq 8 \\
		\hspace{1.6cm} x,y \leq 0
%(x*, y*) = (17.4545,10.90909). f1=85.0909
\end{array}
\end{equation}

} %end footnotesize
\end{enumerate}


The second part of the experiments consists in solving the 
test-collection (SMD1 to SMD6) proposed in \cite{Sinha12}.
%
For those tests, the upper and the lower level functions 
$F(x,y)$ and $f(x,y)$, respectively, are split into three components, as
%
\begin{eqnarray}
F(x,y) = F1(x_1) + F2(y_1) + F3(x_2, y_2)\\
f(x,y) = f1(x_1,x_2) + f2(y_1) + f3(x_2, y_2)
\end{eqnarray}
where
$x=(x_1,x_2)$ and $y=(y_1,y_2)$.
The dimensions of the variables are set to 
$dim(x_1)=p$, $dim(x_2)=r$, $dim(y_1)=q$ and
$dim(y_2)=r$. For SMD6, $dim(y_1)=q+s$.
%where $p,q,r$ and $s$ values define a
%specific problem.


The test problems are described as:

%--------------------------------------

\begin{flushleft}
\begin{tabular}{ll}
SMD1: 	& $F1 = \sum_{i=1}^{p} (x_{1}^{i})^2 $ \\
		& $F2 = \sum_{i=1}^{q} (y_{1}^{i})^2 $ \\
		& $F3 = \sum_{i=1}^{r} (x_{2}^{i})^2 + \sum_{i=1}^{r} (x_{2}^{i} - \tan y_{2}^{i})^2 $\\ 
		& $f1 = \sum_{i=1}^{p} (x_{1}^{i})^2 $\\
		& $f2 = \sum_{i=1}^{q} (y_{1}^{i})^2 $\\
		& $f3 = \sum_{i=1}^{r} (x_{2}^{i} - \tan y_{2}^{i})^2 $\\
\end{tabular}
\end{flushleft}
%
\noindent The ranges of the variables are:
\\
$x_{1}^{i} \in [-5,10], \forall i \in \{1,2,...,p\} $\\
$x_{2}^{i} \in [-5,10], \forall i \in \{1,2,...,r\} $\\
$y_{1}^{i} \in [-5,10], \forall i \in \{1,2,...,q\} $\\
$y_{2}^{i} \in (\frac{-\pi}{2},\frac{\pi}{2}), \forall i \in \{1,2,...,r\} $\\
\\
\noindent The optimal values are $x=0$ and $y$ given by:
$y_{1}^{i} = 0, \forall i \in \{1,2,...,q\}$ and
$y_{2}^{i} = \tan^{-1}x_{2}^{i}, \forall i \in \{1,2,...,r\}$.
The upper and lower level functions are equal to 0 at the optimum.


%--------------------------------------

\begin{flushleft}
\begin{tabular}{ll}
SMD2: 	& $F1 = \sum_{i=1}^{p} (x_{1}^{i})^2 $ \\
		& $F2 = -\sum_{i=1}^{q} (y_{1}^{i})^2 $ \\
		& $F3 = \sum_{i=1}^{r} (x_{2}^{i})^2 - \sum_{i=1}^{r} (x_{2}^{i} - \log y_{2}^{i})^2 $ \\
		& $f1 = \sum_{i=1}^{p} (x_{1}^{i})^2 $\\
		& $f2 = \sum_{i=1}^{q} (y_{1}^{i})^2 $\\
		& $f3 = \sum_{i=1}^{r} (x_{2}^{i} - \log y_{2}^{i})^2 $\\
\end{tabular}
\end{flushleft}
%
\noindent The ranges of the variables are:
\\
$x_{1}^{i} \in [-5,10], \forall i \in \{1,2,...,p\} $\\
$x_{2}^{i} \in [-5,1], \forall i \in \{1,2,...,r\} $\\
$y_{1}^{i} \in [-5,10], \forall i \in \{1,2,...,q\} $\\
$y_{2}^{i} \in (0,e], \forall i \in \{1,2,...,r\} $\\
\\
\noindent The optimal values are $x=0$ and $y$ given by:
$y_{1}^{i} = 0, \forall i \in \{1,2,...,q\}$ and
$y_{2}^{i} = \log^{-1}x_{2}^{i}, \forall i \in \{1,2,...,r\}$.
The upper and lower level functions are equal to 0 at the optimum.


%--------------------------------------

\begin{flushleft}
\begin{tabular}{ll}
SMD3:	& $F1 = \sum_{i=1}^{p} (x_{1}^{i})^2 $ \\
		& $F2 = \sum_{i=1}^{q} (y_{1}^{i})^2 $ \\ 
		& $F3 = \sum_{i=1}^{r} (x_{2}^{i})^2 + \sum_{i=1}^{r} ((x_{2}^{i})^2 - \tan y_{2}^{i})^2 $ \\
		& $f1 = \sum_{i=1}^{p} (x_{1}^{i})^2 $ \\
		& $f2 = q + \sum_{i=1}^{q} ( (y_{1}^{i})^2 - \cos 2 \pi y_{1}^{i} ) $ \\
		& $f3 = \sum_{i=1}^{r} ( (x_{2}^{i})^2 - \tan y_{2}^{i} )^2 $ \\
\end{tabular}
\end{flushleft}
%
\noindent The ranges of the variables are:
\\
$x_{1}^{i} \in [-5,10], \forall i \in \{1,2,...,p\} $\\
$x_{2}^{i} \in [-5,10], \forall i \in \{1,2,...,r\} $\\
$y_{1}^{i} \in [-5,10], \forall i \in \{1,2,...,q\} $\\
$y_{2}^{i} \in (\frac{-\pi}{2},\frac{\pi}{2}), \forall i \in \{1,2,...,r\} $\\
\\
\noindent The optimal values are $x=0$ and $y$ given by: 
$y_{1}^{i} = 0, \forall i \in \{1,2,...,q\}$ and
$y_{2}^{i} = \tan^{-1}x_{2}^{i}, \forall i \in \{1,2,...,r\}$.
The upper and lower level functions are equal to 0 at the optimum.


%--------------------------------------

\begin{flushleft}
\begin{tabular}{ll}
SMD4:	& $F1 = \sum_{i=1}^{p} (x_{1}^{i})^2 $ \\
		& $F2 = - \sum_{i=1}^{q} (y_{1}^{i})^2 $ \\
		& $F3 = \sum_{i=1}^{r} (x_{2}^{i})^2 - \sum_{i=1}^{r} ( |x_{2}^{i}| - \log (1 + y_{2}^{i}) )^2 $ \\
		& $f1 = \sum_{i=1}^{p} (x_{1}^{i})^2 $\\
		& $f2 = q + \sum_{i=1}^{q} ( (y_{1}^{i})^2 - \cos 2 \pi y_{1}^{i} ) $\\
		& $f3 = \sum_{i=1}^{r} ( |x_{2}^{i}| - \log (1+y_{2}^{i}) )^2 $\\
\end{tabular}
\end{flushleft}
%
\noindent The ranges of the variables are:
\\
$x_{1}^{i} \in [-5,10], \forall i \in \{1,2,...,p\} $\\
$x_{2}^{i} \in [-1,1], \forall i \in \{1,2,...,r\} $\\
$y_{1}^{i} \in [-5,10], \forall i \in \{1,2,...,q\} $\\
$y_{2}^{i} \in [0,e], \forall i \in \{1,2,...,r\} $\\
\\
\noindent The optimal values are $x=0$ and $y$ given by:
$y_{1}^{i} = 0, \forall i \in \{1,2,...,q\}$ and
$y_{2}^{i} = \log^{-1}|x_{2}^{i}|-1, \forall i \in \{1,2,...,r\}$.
The upper and lower level functions are equal to 0 at the optimum.


%--------------------------------------

\begin{flushleft}
\begin{tabular}{ll}
SMD5:	& $F1 = \sum_{i=1}^{p} (x_{1}^{i})^2 $ \\
		& $F2 = - \sum_{i=1}^{q-1} ( ( y_{1}^{i+1} - (y_{1}^{i})^2 )^2 + ( y_{1}^{i} - 1 )^2 ) $ \\
		& $F3 = \sum_{i=1}^{r} (x_{2}^{i})^2 - \sum_{i=1}^{r} ( |x_{2}^{i}| - (y_{2}^{i})^2 )^2 $ \\
		& $f1 = \sum_{i=1}^{p} (x_{1}^{i})^2 $ \\
		& $f2 = \sum_{i=1}^{q-1} ( ( y_{1}^{i+1} - (y_{1}^{i})^2 )^2 + ( y_{1}^{i} - 1 )^2 ) $ \\
		& $f3 = \sum_{i=1}^{r} ( |x_{2}^{i}| - (y_{2}^{i})^2 )^2 $ \\
\end{tabular}
\end{flushleft}
%
\noindent The ranges of the variables are:
\\
$x_{1}^{i} \in [-5,10], \forall i \in \{1,2,...,p\} $\\
$x_{2}^{i},y_{2}^{i} \in [-5,10], \forall i \in \{1,2,...,r\} $\\
$y_{1}^{i} \in [-5,10], \forall i \in \{1,2,...,q\} $\\
%$y_{2}^{i} \in [-5,10], \forall i \in \{1,2,...,r\} $\\
\\
\noindent The optimal values are $x=0$ and $y$ given by:
$y_{1}^{i} = 1, \forall i \in \{1,2,...,q\}$ and
$y_{2}^{i} = \sqrt{|x_{2}^{i}|}, \forall i \in \{1,2,...,r\}$.
The upper and lower level functions are equal to 0 at the optimum.


%--------------------------------------
\begin{flushleft}
\begin{tabular}{ll}
SMD6:	& $F1 = \sum_{i=1}^{p} (x_{1}^{i})^2 $ \\
		& $F2 = - \sum_{i=1}^{q} ( y_{1}^{i} )^2 + \sum_{i=q+1}^{q+s} ( y_{1}^{i} )^2 $ \\
		& $F3 = \sum_{i=1}^{r} (x_{2}^{i})^2 - \sum_{i=1}^{r} ( x_{2}^{i} - y_{2}^{i} )^2 $ \\
		& $f1 = \sum_{i=1}^{p} (x_{1}^{i})^2 $ \\
		& $f2 = \sum_{i=1}^{q} ( y_{1}^{i} )^2 + \sum_{i=q+1}^{q+s-1} ( y_{1}^{i+1} - y_{1}^{i} )^2 $ \\
		& $f3 = \sum_{i=1}^{r} ( x_{2}^{i} - y_{2}^{i} )^2 $ \\
\end{tabular}
\end{flushleft}
%		
\noindent The ranges of the variables are:
\\
$x_{1}^{i} \in [-5,10], \forall i \in \{1,2,...,p\} $\\
$x_{2}^{i},y_{2}^{i} \in [-5,10], \forall i \in \{1,2,...,r\} $\\
$y_{1}^{i} \in [-5,10], \forall i \in \{1,2,...,q+s\} $\\
%$y_{2}^{i} \in [-5,10], \forall i \in \{1,2,...,r\} $\\
\\
\noindent The optimal values are $x=0$ and $y$ given by:
$y_{1}^{i} = 0, \forall i \in \{1,2,...,q\}$ and
$y_{2}^{i} = x_{2}^{i}, \forall i \in \{1,2,...,r\}$.
The upper and lower level functions are equal to 0 at the optimum.


For the test-collection (SMD1 to SMD6) the instances considered 
have 10 dimensions and correspond to setting $p=3$, $q=2$, and $r=2$ 
for problems SMD1 to SMD5, and $p=3$, $q=1$, $r=2$ and $s=2$ for 
problem SMD6.


%--------------------------------------------------------------------
\subsection{Results}

Table \ref{tab:results} presents the results obtained from solving
the eighteen test-problems, showing the best solutions $(x^*,y^*)$,
and the corresponding values $f^{*}_{1}$ and $f^{*}_{2}$ attained at the upper
and lower levels, respectively, obtained by BlDE and other methods in different sources
of the literature.
%
Table \ref{tab:summary18} summarizes the statistics for BlDE in the 30 independent runs
performed in the same test problems.
The column headers denoted by ``Min'', ``Median'', ``Mean'', ``Max'', and ``std''
denote the minimum, median, average, maximum, and standard deviation values.
%
The number of function evaluations for the upper and lower levels 
were 6000 and 18000000, respectively.


For those test-problems the proposed method was capable of reaching
the optimal solution, or very close to the optimal, in all problems 
tested, and found better solutions for problem 4.
%
As demonstrated in Table \ref{tab:summary18} the algorithm 
presented a stable behavior, since for most problems tested the
algorithm was able to reach the optimal solution in all 
30 executions.

The results for the test-problems SMD1 to SMD6
are reported in Table \ref{tab:resultsDeb}, which
presents the median values of function evaluations for each level, 
and the accuracy achieved for both BlDE and the GA adopted in the paper~\cite{Sinha12}
where the problems were proposed.
%
Table \ref{tab:summaryDeb} summarizes the results obtained by 
BlDE in the SMD's problems.
%
For those tests the number of function evaluations for the upper and 
lower levels were 2400 and 7200000, respectively.


The results described in Table  \ref{tab:resultsDeb},
demonstrate that BlDE generates the optimal solution
for all test problems, and reaches a better accuracy on both levels
in all SMD's problems.
%
For those test problems, the algorithm stability was 
also observed, as shown in Table \ref{tab:summaryDeb} 
by the standard deviation values very close to zero in all problems tested.




\begin{table}[h!]
\centering
\caption{Comparison of best solutions obtained by BlDE and other methods.
A question mark ``?'' indicates a value not provided in the reference.}
\label{tab:results}
\begin{tabular}{p{0.3cm}cccc}
\hline
Pr. & Ref & $(x^*,y^*)$ & $f^{*}_{1}$ & $f^{*}_{2}$ \\
\hline
1	 	& \cite{Shimizu81, Aiyoshi84, Visweswaran96} & (10, 10)       & 100      & 0 \\
		& \cite{Bard98, Rajesh03, Koh07} 			 & (10, 10)	      &	100      & 0 \\
		& \cite{Oduguwa02} 							 & (10.03, 9.969) & 100.58   & 0.001 \\
		& \cite{Zhu06} 								 & (10.01, 9.93)  & 100.2050 & 0.0169 \\
		& BlDE 										 & (10, 10) 	  & 100      & 0 \\
\hline	
2	 	& \cite{Shimizu81, Visweswaran96, Zhu06} & (20, 5, 10, 5) 		& 225     & 100 \\
		& BlDE 									 & (19.9999, 5.00004,   & 224.988 & 99.9987 \\
		&      									 &  10,  4.99941)       &         &        \\
\hline	
3       & \cite{Candler82, Bard82, Bard98}	& (0, 0.9, 0, 0.6, 0.4) 		& 29.2 	  & -3.2 \\
%		& \cite{Onal93} 				 	& (?, 0.9, ?, 0.6, 0.4) 	 	& 7.3 	  & ? \\		
		& \cite{Wang03} 					& (0, 0.899, 0, 0.06, 0.394) 	& 29.172  & -3.186 \\
		& \cite{GuangMin05} 				& (0, 0.898, 0, 0.599, 0.399)	& 29.148  & -3.193 \\
		& \cite{Zhu06} 						& (0, 0.9, 0, 0.6, 0.3999)   	& 29.198  & -3.2 \\
		& \cite{Wang07} 					& (0, 0.8993, 0, 0.5995, 0.3981)& 29.1709 & -3.1805 \\
		& BlDE 								& (0, 0.899998,  2.15e-6,  		&  29.2   & -3,2\\
		&      								&  0.600001, 0.4000001)      	&         &        \\
\hline
4		& \cite{Bard82, GuangMin05} 	& (1, 0, 0.5, 1) 				& 1.75 	  & 0 \\
%		& \cite{Onal93} 				& (1, ?, 0.5, 1) 				& 1.75 	  & ? \\
		& \cite{Wang07}					& (0.9993, 0, 0.4993, 1.0008) 	& 1.7488  & -0.0028 \\
		& BlDE 							& (2, 0, 1.5, 0)				& 3.25    & 4 \\
\hline
5		& \cite{Aiyoshi84, Bard98, Li12}	& (25, 30, 5, 10) 			& 5 		& 0 \\
		& \cite{Ishizuka92, Visweswaran96} 	& (0, 0, -10, -10) 			& 0 		& 200 \\
		& \cite{Facchinei99} 				& (25.00125, 30, ?, ?) 		& 4.999375 	& ? \\
		& \cite{Zhu06} 						& (24.87, 29.98, 5.1, 10.31)& 3.47 		& 0.1618 \\				
		& \cite{Koh07} 						& (0, 30, 10, -10) 			& 0 		& 100 \\
		& BlDE 								& (0, 0, -10, -10) 			& 0 		& 200 \\
\hline
6		& \cite{Bard88, Bard98} 					& (1.778, 2.333) 	& 42.5 	& -4.445 \\
		& \cite{Visweswaran96, Rajesh03, Koh07} 	& (1, 0) 			& 17 	& 1 \\
		& BlDE 										& (1, 0) 			& 17 	& 1 \\						
\hline
7		%& \cite{Bard88} 			& (1.45, 0.95, 1.88, 0.64) 	& -12.0629	& 11 \\
		& \cite{Facchinei99} 		& (0, 2, ?, ?) 		 		& -12.67871 & ? \\
		& BlDE 						& (0, 1.99996,       		& -12.6799  & -1.01557\\
		&      						& 1.87549, 0.906606) 		&           & \\
\hline
8		& \cite{Anandalingam90, Radjef10} 	& (16, 11) 			& 49 	  & -17 \\	
		& \cite{Wang03} 					& (15.959, 10.972) 	& 48.875  & -16.957 \\
		& \cite{GuangMin05} 				& (15.9972, 10.9945)& 48.9806 & -16.9863 \\
		& \cite{Wang07} 					& (15.9966, 10.9933)& 48.9764 & -16.9833 \\
		& BlDE 								& (16, 11) 			& 49 	  & -17 \\						
\hline
9	 	& \cite{Savard94, Bard98} 	& (1.889, 0.889, 0) 	& -1.21 		& 7.61 \\
		& \cite{Oduguwa02} 			& (?, ?, ?) 			& 3.57 			& 2.4 \\		
		& \cite{Zhu06} 				& (1.87, 0.89, 0) 		& -1.2031 		& 7.5927 \\
		& BlDE 						& (1.88889, 0.888889, 0)& -1.209876568 	& 7.61728 \\ 
		%por arredondamento o valor de F fica −1,209876568, o calculado foi -1.40741
\hline
10		& \cite{Falk95, Facchinei99, Koh07} & (0.5, 0.5, 0.5, 0.5) & -1 & 0 \\
		& BlDE 								& (0.5, 0.5, 0.5, 0.5) & -1 & 0 \\		
\hline
11		& \cite{Shimizu95, Bard98} 			& (11.25, 5)	& 2250 & 197.75  \\
		& BlDE 								& (11.25, 5) 	& 2250 & 197.754 \\		
\hline
12		& \cite{Bard98, Radjef10} 	& (4, 4) & -12 & 4 \\ % (pg.197)
		& BlDE 						& (4, 4) & -12 & 4 \\
\hline
13)		& \cite{Bard98, Rajesh03, Zhu06} 	& (0.889, 2.222) 		& 3.111	  & -6.667 \\ %(pg.263)
		& BlDE 								& (0.888889, 2.22222) 	& 3.11111 & -6.66667\\
\hline
14		& \cite{Bard98}  	& (1, 0) 			& 1 	& 0 \\%(10.02, 0.82) & 81.33 & ? \\ %(pg.306)
		& \cite{Oduguwa02} 	& (10.04, 0.1429) 	& 82.44 & -0.271 \\
		& \cite{Koh09} 		& (?, ?) 			& 81.33 & 0.34 \\
		& BlDE 				& (1, 0) 			& 1 	& 0\\
\hline
15		& \cite{Oduguwa02} 	& (?, ?, ?) 				 & 1000 	& 1 \\
		& \cite{Li12} 		& (1, 0, 1) 				 & 1000 	& 1 \\
		& BlDE 				& (0, 0.999657, 0.000342909) & 999.657 	& 1 \\
\hline
16		& \cite{Rajesh03} 	& (1, 3) & 5 & 4 \\ % (Ex.2)
		& \cite{Koh07} 		& (4, 3) & 2 & 4 \\
		& BlDE 				& (1, 3) & 5 & 4 \\		
\hline
17		& \cite{Rajesh03, Koh07} 	& (3, 5) & 9 & 0 \\ % (Ex.3)
		& BlDE 						& (3, 5) & 9 & 0 \\
\hline
18		& \cite{Rajesh03} 	& (17.4545, 10.90909) 	& 85.0909 	& ? \\ % (Ex.4)		
		& BlDE 				& (17.4545, 10.9091)	& 85.0909  	& -50.1818 \\
\hline	
\end{tabular}
\end{table}
%
%
\begin{table}[h!]
\centering
\caption{Summary of results obtained by BlDE for test-problems 1 to 18.}
\label{tab:summary18}
\begin{tabular}{p{0.3cm}ccccccc}
\hline
Pr. & Level & Min           & Median       & Mean         & Max           & std \\\hline
 1      & UL    & 100         & 100        & 100        & 100         & 0   \\   
        & LL    & 0           & 0          & 0          & 0           & 0   \\   
\hline
 2      & UL    & 224.9880    & 224.9920   & 224.9919   & 224.9940    & 1.6e-03   \\   
        & LL    &  99.9808    &  99.9976   & 99.9967    &  99.9994    & 3.4e-03   \\   
\hline
 3      & UL    & 16          & 29.2       & 28.7600    & 29.2002     & 2.4   \\   
        & LL    & -6.5        & -3.2       & -3.3100    & -3.1999     & 6.0e-01   \\   
\hline
 4      & UL    & 3.25        & 3.6142     & 3.6247     & 3.9892      & 2.6e-01   \\   
        & LL    & -1.9141     &  0.9842    &  0.9863    &  4          & 2.1   \\   
\hline
 5      & UL    & -7.38e-08   &  0         & -2.64e-09  &  0          & 1.3e-08   \\   
        & LL    & 100         & 200        & 193.3333   & 200         & 2.53e+01   \\   
\hline
 6      & UL    & 16.8612     & 17         & 16.995417  & 17.0023     & 2.5e-02   \\   
        & LL    & 0.9939      & 1          & 0.9997993  & 1           & 1.0e-03   \\   
\hline
 7      & UL    & -12.6905    & -12.6795   & -12.6800   & -12.6792    & 20e-03   \\   
        & LL    & -1.0156     & -1.01545   & -1.0153    & -1.0133     & 4.1e-04   \\   
\hline
 8      & UL    & 49          & 49         & 49         & 49          & 0   \\   
        & LL    & -17         & -17        & -17        & -17         & 0   \\   
\hline
 9      & UL    & -1.4074     & -1.4074    & -1.4074    & -1.4074     & 0   \\   
        & LL    & 7.6172      & 7.6172     & 7.6172     & 7.6172      & 0   \\   
\hline
 10     & UL    & -1          & -1         & -1         & -1          & 1.8e-06   \\   
        & LL    & 1.09e-12    & 4.00e-12   & 1.21e-11   & 1.44e-10    & 2.9e-11   \\   
\hline
 11     & UL    & 2250        & 2250       & 2250       & 2250        & 0   \\   
        & LL    & 197.754     & 197.754    & 197.754    & 197.754     & 0   \\   
\hline
 12     & UL    & -12         & -12        & -12        & -12         & 0   \\   
        & LL    & 4           & 4          & 4          & 4           & 0   \\   
\hline
 13     & UL    & 3.1111      & 3.1111     & 3.1111     & 3.1111      & 0   \\   
        & LL    & -6.6666     & -6.6666    & -6.6666    & -6.6666     & 0   \\   
\hline
 14     & UL    & 1           & 1          & 1          & 1           & 0   \\   
        & LL    & 0           & 0          & 0          & 0           & 0   \\   
\hline
 15     & UL    & 980.1060    & 993.6590   & 993.2020   & 999.6570    & 4.6   \\   
        & LL    & 1           & 1          & 1          & 1           & 0   \\   
\hline
 16     & UL    & 5           & 5          & 5          & 5           & 0   \\   
        & LL    & 4           & 4          & 4          & 4           & 0   \\   
\hline
 17     & UL    & 9           & 9          & 9          & 9           & 0   \\   
        & LL    & 0           & 0          & 0          & 0           & 0   \\   
\hline
 18     & UL    & 79.8455     & 85.0909    & 84.4802    & 85.0909     & 1.5   \\   
        & LL    & -50.1818    & -50.1818   & -49.9680   & -48.3459    & 5.2e-01   \\   

\hline
\end{tabular}
\end{table}


\begin{table}[h!]
\centering
\caption{Median values for function evaluations (FE) and 
accuracy (Acc) for the upper and lower levels.}
\label{tab:resultsDeb}
\begin{tabular}{c|cccc|cc}
\hline
	    & \multicolumn{4}{c}{GA \cite{Sinha12}} & \multicolumn{2}{c}{BlDE} \\
Pr. & UL FE & LL FE & UL Acc & LL Acc   & UL Acc    & LL Acc \\
\hline
%SMD1 & 2644 & 1724241 & 0.000034 & 0.000016 & 3.491e-06 & 1.935e-06 \\
SMD1 & 2644 & 1724241 & 3.4e-05 & 1.6e-05  & 3.491e-06 & 1.935e-06 \\
%SMD2 & 2404 & 1568099 & 0.000005 & 0.000005 & 1.294e-06 & 6.514e-07\\
SMD2 & 2404 & 1568099 & 5.0e-06  & 5.0e-06 & 1.294e-06 & 6.514e-07\\
%SMD3 & 2338 & 1483884 & 0.000059 & 0.000026 & 4.096e-06 & 2.917e-06\\
SMD3 & 2338 & 1483884 & 5.9e-05 & 2.6e-05 & 4.096e-06 & 2.917e-06\\
%SMD4 & 1720 & 1187981 & 0.000026 & 0.000027 & -2.296e-05& 5.140e-05\\
SMD4 & 1720 & 1187981 & 2.6e-05 & 2.7e-05 & 2.296e-05& 5.140e-05\\
%SMD5 & 3010 & 2093391 & 0.000004 & 0.000003 & 1.581e-06 & 1.379e-06\\
SMD5 & 3010 & 2093391 & 4.0e-06 & 3.0e-06 & 1.581e-06 & 1.379e-06\\
%SMD6 & 3212 & 2429352 & 0.000145 & 0.000071 & 3.470e-06 & 2.067e-06\\
SMD6 & 3212 & 2429352 & 1.45e-04 & 7.1e-05 & 3.470e-06 & 2.067e-06\\
\hline
\end{tabular}
\end{table}


\begin{table}[h!]
\centering
\caption{Summary of results obtained by BlDE for test-collection SMD1 to SMD6.}
\label{tab:summaryDeb}
\begin{tabular}{p{0.3cm}ccccccc}%p{0.4cm}ccccccc
\hline
Pr. & Level& Min              & Median           & Mean             & Max              & std \\\hline
%\multicolumn{6}{c}{SMD1}\\\hline
SMD1 & UL   & 3.73e-07       & 3.49e-06       & 3.83e-06       & 1.23e-05       & 3.05e-06   \\   
     & LL   & 2.46e-07       & 1.93e-06       & 2.59e-06       & 8.00e-06       & 2.36e-06 \\   
\hline
%\multicolumn{6}{c}{SMD2}\\\hline
SMD2 & UL   & 2.32e-07       & 1.29e-06       & 1.68e-06       & 8.18e-06       & 1.50e-06   \\   
     & LL   & 1.57e-07       & 6.51e-07       & 9.43e-07       & 2.74e-06       & 7.09e-07   \\   
\hline
%\multicolumn{6}{c}{SMD3}\\\hline
SMD3 & UL   & 1.06e-06       & 4.09e-06       & 4.78e-06       & 9.90e-06       & 2.56e-06   \\   
     & LL   & 4.59e-07       & 2.91e-06       & 3.93e-06       & 1.62e-05       & 3.62e-06   \\   
\hline
%\multicolumn{6}{c}{SMD4}\\\hline
SMD4 & UL   & -4.01e-04      & -2.29e-05      & -4.76e-05       & 1.24e-06      & 8.14e-05   \\   
     & LL   & 2.4e-07        & 5.10e-05         & 1.0e-04       & 7.76e-04      & 1.60e-04   \\   
\hline
%\multicolumn{6}{c}{SMD5}\\\hline
SMD5 & UL   & 2.97e-07       & 1.58e-06       & 2.93e-06       & 1.10e-05       & 2.67e-06   \\   
     & LL   & 1.61e-07       & 1.37e-06       & 2.06e-06       & 6.50e-06       & 1.93e-06   \\   
\hline
%\multicolumn{6}{c}{SMD6}\\\hline
SMD6 & UL   & 3.90e-07       & 3.47e-06       & 3.77e-06       & 1.24e-05       & 3.08e-06   \\   
     & LL   & 2.81e-07       & 2.06e-06       & 2.62e-06       & 8.07e-06       & 2.39e-06   \\  
\hline
\end{tabular}
\end{table}

\subsection{Discussion}
It is important to mention that different DE variants
as well as different parameter values for F, CR, population size, and number of generations could be
adopted in the different levels of the optimization.
%
In fact, a more efficient algorithm would probably arise by adapting the values of the DE
parameters as well as by closely monitoring the convergence process in both populations
in order to save function evaluations by stopping the iterations earlier.

Furthermore, since the algorithmic idea proposed used two DE algorithms
independently, other evolutionary methods
could be incorporated in different levels of the BLP, e.g., by using 
DE on the upper level and an ant colony algorithm on the lower level, 
or vice-versa.
%
The implementation of different algorithms in different levels of the BLP could
be interesting when, for instance, the lower level problem can be more efficiently solved by an existing 
problem specific technique.

%--------------------------------------------------------------------
\section{Conclusion}\label{sec:conclusion}
%--------------------------------------------------------------------


In this paper a simple method using the differential evolution technique 
was proposed for solving general bilevel programming problems in continuous variables.
%
The algorithm uses two DE algorithms, one responsible for 
optimizing the upper level problem, and another for optimizing the 
lower level problem.
%
The algorithm was tested in a variety of test-problems taken from the
literature  including linear, nonlinear, constrained, and 
unconstrained optimization problems.
%
Although no tuning procedure for the DE parameters has been attempted
in the experiments, the results demonstrated that the proposed algorithm 
successfully solved the test problems considered.

For future work, it is intended to extend this research in order to propose ways of reducing
the number of function evaluations required by the algorithm, specially in the lower level
problem.
%



% conference papers do not normally have an appendix
% use section* for acknowledgement
%--------------------------------------------------------------------
\section*{Acknowledgment}
%--------------------------------------------------------------------

The authors would like to thank the support from CNPq 
(grants 141519/2010-0, 140785/2009-4, 308317/2009-2).

%\clearpage

\bibliographystyle{IEEEtran}
\bibliography{references}

\end{document}


